\documentclass{article}

% Use fontspec for custom fonts
\usepackage[absolute,overlay]{textpos}
\usepackage{graphicx}
\usepackage{microtype} % For advanced spacing control
\usepackage{calc} % For width calculation
\usepackage{fontspec}
\usepackage{tikz}
\usepackage{microtype}
\setmainfont{THSarabun.ttf}[
    BoldFont = THSarabun-Bold.ttf
]

\usepackage[normalem]{ulem}
\usepackage{amsmath} % Required for \stackrel
\usepackage{xcolor}

% Strict margin settings
\usepackage[a4paper, left=3cm, right=2cm, top=1.5cm, bottom=2cm]{geometry}

% Ensure no text exceeds margins
\sloppy

% Line spacing
\usepackage{setspace}
\setstretch{1.9}

% Thai numbering and proper UTF-8 handling
\usepackage[thai,english]{babel}

\pagenumbering{gobble}

% Define a global Y position counter (starting at a specified position)
\newdimen\CurrentYPosition
\CurrentYPosition=6.15cm % Initial starting point for the first paragraph

% Define the spacing between paragraphs (6pt converted to cm for compatibility)
\newcommand{\ParagraphSpacing}{0.21cm} % Adjust as needed

% Custom paragraph marker counter
\newcounter{mainparagraph} % Counter for automatic numbering

% Thai numbers mapping
\newcommand{\thainumber}[1]{%
    \ifcase#1 ๐\or ๑\or ๒\or ๓\or ๔\or ๕\or ๖\or ๗\or ๘\or ๙\else#1\fi%
}

% Display Thai numbers for the counter
\renewcommand{\themainparagraph}{\thainumber{\value{mainparagraph}}}

% Paragraph marker command (acts as a number counter)
\newcommand{\parmark}{\refstepcounter{mainparagraph}\themainparagraph.}




\newcommand{\TabOne}{\hspace*{1.73cm}}
\newcommand{\TabTwo}{\hspace*{2.42cm}}
\newcommand{\TabThree}{\hspace*{3.32cm}}

\newcommand{\TabAft}{\hspace*{0.25cm}}
\newcommand{\TabAftTwo}{\hspace*{0cm}}
% Special newline with extra space before (like Word's "Before 6pt")
\newcommand{\newpar}{\vspace{0.21cm}}

% Command for Thai Distributed Text
\newcommand{\thaidistributed}[1]{%
    \spaceskip=0.5em plus 1em minus 0.3em % Controls inter-word spacing
    \xspaceskip=0.5em % Controls extra space between characters
    #1
}






\begin{document}


% Set font size to 20pt
\fontsize{16pt}{16pt}\selectfont


\begin{textblock*}{15cm}(2.3cm, 1.15cm)  % (width, x-pos, y-pos)
\includegraphics[height=1.5cm]{Bird.png} % Ensure path is correct
\end{textblock*}

\begin{textblock*}{15cm}(8.85cm, 1.95cm)  % (width, x-pos, y-pos)
{\fontsize{29pt}{29pt}\textbf{บันทึกข้อความ}}
\end{textblock*}



\begin{textblock*}{15cm}(2.8cm, 3.1cm)  % (width, x-pos, y-pos)
\noindent
{\fontsize{20pt}{24pt}\selectfont\textbf{ส่วนราชการ}} 
\hspace{-0.2cm} 
{\raisebox{-0.5ex}{\makebox[\dimexpr\linewidth-0.78cm\relax]{\leaders\hbox to 0.1em{\hss{\fontsize{10pt}{10pt}\selectfont .}\hss}\hfill}}} \\[1.5ex]
\end{textblock*}


\begin{textblock*}{15cm}(2.8cm, 3.9cm)  % (width, x-pos, y-pos)
\noindent
{\fontsize{20pt}{24pt}\selectfont\textbf{ที่}} 
\hspace{-0.2cm} 
{\raisebox{-0.5ex}{\makebox[\dimexpr\linewidth-7.3cm\relax]{\leaders\hbox to 0.1em{\hss{\fontsize{10pt}{10pt}\selectfont .}\hss}\hfill}}} \\[1.5ex]
\end{textblock*}

\begin{textblock*}{15cm}(10.9cm, 3.9cm)  % (width, x-pos, y-pos)
\noindent
{\fontsize{20pt}{24pt}\selectfont\textbf{วันที่}} 
\hspace{-0.25cm} 
{\raisebox{-0.5ex}{\makebox[\dimexpr\linewidth-7.48cm\relax]{\leaders\hbox to 0.1em{\hss{\fontsize{10pt}{10pt}\selectfont .}\hss}\hfill}}} \\[1.5ex]
\end{textblock*}

\begin{textblock*}{16cm}(2.8cm, 4.7cm)  % (width, x-pos, y-pos)
\noindent
{\fontsize{20pt}{24pt}\selectfont\textbf{เรื่อง}} 
\hspace{-0.15cm} 
{\raisebox{-0.5ex}{\makebox[\dimexpr\linewidth-0.38cm\relax]{\leaders\hbox to 0.1em{\hss{\fontsize{10pt}{10pt}\selectfont .}\hss}\hfill}}} \\[1.5ex]
\end{textblock*}



\begin{textblock*}{10cm}(5cm, 3.16cm)
    {\fontsize{16pt}{16pt}{กฝพ.ศซบ.ทอ.(ผพร.โทร.๒-๓๗๖๐)}}
\end{textblock*}


\begin{textblock*}{15cm}(3.5cm, 4.85cm)
    {\fontsize{16pt}{16pt}{สรุปผลการตรวจสอบการรักษาความมั่นคงปลอดภัยทางไซเบอร์เชิงเทคนิคของ บน.๙}}
\end{textblock*}

\begin{textblock*}{5cm}(2.2cm, 5.7cm)
    {\fontsize{16pt}{16pt}{เรียน รอง ผอ.ศซบ.ทอ}}
\end{textblock*}


\vspace*{4.3cm}

\TabOne\parmark\TabAft ตามหนังสือ ศซบ.ทอ.ที่ กห ๐๑๒๓.๔/๕๖๗๘ ลง ๒๙ ธ.ค.๖๑ ได้ขอดำเนินการประเมินและ ตรวจสอบการรักษาความมั่นคงปลอดภัยทางไซเบอร์เชิงเทคนิครวมถึงการสรุปผลให้ความรู้และกำหนดแนวทาง ในการป้องกันระบบสารสนเทศให้เกิดความมั่นคงปลอดภัยให้กับ บน.๙ ระหว่าง ๑๒ - ๓๑ ม.ค.๖๒ นั้น

\newpar
\TabOne\parmark\TabAft ศซบ.ทอ.ดำเนินการตามข้อ ๑ เรียบร้อยแล้ว ดังนี้

\TabTwo ๒.๑ การตรวจสอบการรักษาความมั่นคงปลอดภัยทางไซเบอร์เชิงเทคนิค เพื่อเสริมสร้าง
ขีดความสามารถการปฏิบัติการไซเบอร์ โดยใช้แนวทางของมาตรฐาน ISO/IEC 27001:2022 \scalebox{.8}[1.0]{Information security,} \\cybersecurity and privacy protection \scalebox{0.95}[1.0]{ - Information security management systems - Requirements}
และมาตรฐาน National Institute of Standards and Technology (NIST) 800-115 :\scalebox{1.1}[1.0]{ Technical Guide}\\to Information Security Testing and Assessment ซึ่งเป็นมาตรฐานสากลในการรักษาความมั่นคงปลอดภัย\\ทางไซเบอร์ในการตรวจสอบฯ


\TabTwo ๒.๒ ผล\scalebox{1.05}[1.0]{การตรวจสอบการรักษาความมั่นคงปลอดภัยทางไซเบอร์เชิงเทคนิคของ} บน.๙ พบช่องโหว่ที่สำคัญ ดังนี้

\TabThree ๒.๒.๑	อุปกรณ์ที่ไม่มีการตั้งรหัสผ่าน (No Password)







\begin{textblock*}{15cm}(4.2cm, 27cm)
\begin{flushright}
    {\fontsize{16pt}{19pt} ๒.๑.๓ ติดตั้ง...}
\end{flushright}
\end{textblock*}

\end{document}
